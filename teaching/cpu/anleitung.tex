\documentclass[a4paper]{kinet-intern}

\department{FS Informatik}
\author{jat, ros}
\title{CPU-Rollenspiel}

\begin{document}

\maketitle

\newcommand{\voice}[1]{\textbf{\dq{}#1\dq{}}}

% ==============================================================================
\section*{Hinweise für Lehrpersonen}
\subsection*{Von Neumann}
Es ist empfehlenswert, zuerst die Von-Neumann-Architektur zu behandeln.

\subsection*{Rollen}
Vorgesehen sind 5-6 Rollen. Die Rolle des Zeitgebers kann gestrichen werden, ohne dass die Funktionalität leidet, sie eignet sich aber gut für die abschliessende Diskussion. Die Rolle des Bildschirms (7. Rolle), ist ausschliesslich passiv, daher wird diese nicht von SuS gespielt, es liegt einfach ein Blatt mit dem Titel \dq{}Bildschirm\dq{} auf.

Die Rollen (siehe unten) sollen ausgedruckt und den entsprechenden SuS verteilt werden. Der Decodierer erhält zusätzlich noch den einfachen Befehlssatz (resp. in einer zweiten Phase den erweiterten Befehlssatz). Das aufgelistete Programm direkt unter dem Befehlssatz soll nicht verteilt werden, es dient lediglich der Übersicht für die Lehrperson.

\subsection*{Zusätzliche Blätter}
Die Blätter mit dem Speicherinhalt (für beide Programme) sowie die fast leeren Blätter, die während der Laufzeit beschrieben werden, sind separat als odt und pdf vorhanden.

\subsection*{Programm in der erweiterten Sprache}
Das Programm zeigt die geraden Zahlen von 1 bis 10 an. Dazu reichen die Speicherstellen von 1 bis 9 nicht mehr aus. Auch die Speicherstelle 0 kann nicht verwendet werden, da ja eine 0 als zweite Ziffer in einem Befehl stets bedeutet, dass keine Speicherstelle involviert ist. Deshalb wird als 10. Speicherstelle die Speicherstelle A verwendet. Dies, damit die Speicherstelle mit nur einer \dq{}Ziffer\dq{} geschrieben werden kann.

Das Programm dauert einige Zeit, auch wenn die Schülerinnen und Schüler konzentriert arbeiten. Je nach Stimmung kann während der Laufzeit des Programms auch die Speicherstelle A von 10 auf 8 oder sogar 6 abgeändert werden, um das Programm schneller zu beenden.

\subsection*{Zusatzmaterial}
Für sämtliche Rollen ist unten angegeben, welches Zusatzmaterial (zu den Rollen) steht.
Vorschlag:
\begin{itemize}
	\item Die Blätter für den Adressierer (Speicherinhalte) sowie die restlichen Blätter (CPU-Leiter, Speicherzeiger, Rechenknecht, Zeitgeber, Bildschirm) mit der Funktion 2-auf-1 drucken, so dass die Blätter nur A5 gross sind und weniger Papier gebraucht wird.
	\item Die Blätter für den Decodierer in Streifen schneiden, so dass pro Befehl ein Streifen vorliegt.
\end{itemize}

Die nachfolgenden Blätter mit den Rollen müssen ebenfalls so in Streifen geschnitten werden, dass die Rollen verteilt werden können.

\subsection*{Abschliessende Diskussion}
Hier kann z.B. der Vergleich zu einer richtigen CPU gemacht werden. Der Zeitgeber notiert für jede Runde, wie lange sie dauert. Eine richtige CPU arbeitet mit ca. 3 Milliarden Taktzyklen pro Sekunde...

\pagebreak

\section*{Der CPU-Leiter}

\subsection*{Zusatzmaterial}
\begin{itemize}
	\item Blatt \dq{}CPU-Leiter\dq{} (Register)
\end{itemize}

\subsection*{Fähigkeiten}
\begin{itemize}
	\item ist der Chef, keiner tut etwas ohne seinen Befehl
	\item arbeitet Befehle gemäss untenstehender Checkliste ab
\end{itemize}

\subsection*{Aktionen}
\begin{enumerate}
	\item setzt den Speicherzeiger auf 1:\\
		\voice{Speicherzeiger, setz deinen Wert auf 1!}
	\item gibt dem Zeitgebers einen Impuls:\\
		\voice{Zeitgeber, neue Runde starten!}
	\item fragt den Speicherzeiger nach der aktuellen Adresse:\\
		\voice{Speicherzeiger, wie lautet dein aktueller Wert?}
	\item ruft dem Adressierer die erhaltene Nummer (= X) zu:\\
		\voice{Adressierer, Nummer X!}
	\item befiehlt dem Datenboten, den Speicherinhalt beim Adressierer abzuholen:\\
		\voice{Datenbote, hol die Zahl vom Adressierer zu mir.}
	\item ruft dem Decodierer die 1. Ziffer des erhaltenen Befehls (= Y) zu:\\
		\voice{Decodierer, Nummer Y!}
	\item ruft dem Adressierer die 2. Ziffer des erhaltenen Befehls (= Z) zu:\\
		\voice{Adressierer, Nummer Z!}
	\item führt die Aktion gemäss Befehlssatz aus:
		\vspace{9cm}
	\item befiehlt dem Speicherzeiger, seinen Wert zu erhöhen:\\
		\voice{Speicherzeiger, Zahl erhöhen!}
	\item geht wieder zu Punkt 2
\end{enumerate}


%\vspace{1cm}
\section*{Der Decodierer}

\subsection*{Zusatzmaterial}
\begin{itemize}
	\item Befehlssatz für die einfache (resp. für die erweiterte) Sprache in Streifen geschnitten
\end{itemize}

\subsection*{Fähigkeiten}
\begin{itemize}
	\item decodiert Programm-Befehle, die der CPU-Leiter aus dem Speicher erhält
\end{itemize}

\subsection*{Aktionen}
\begin{enumerate}
	\item wartet, bis er eine Nummer des CPU-Leiters erhält
	\item falls bereits ein Blatt beim CPU-Leiter liegt, nimmer er das Blatt wieder zurück
	\item sucht im Stapel nach dem Blatt mit der entsprechenden Nummer
	\item legt das Blatt auf die leere Stelle beim CPU-Leiter
	\item quittiert die Aktion mit: \voice{Fertig!}
\end{enumerate}


\vspace{1cm}
\section*{Der Adressierer}

\subsection*{Zusatzmaterial}
\begin{itemize}
	\item Speicherinhalte (Programm und Daten)
\end{itemize}

\subsection*{Fähigkeiten}
\begin{itemize}
	\item verwaltet den Speicher für Lese- oder Schreibzugriffe
\end{itemize}

\subsection*{Aktionen}
\begin{enumerate}
	\item wartet, bis er eine Nummer vom CPU-Leiter zugerufen erhält
	\item wenn die zugerufene Zahl 0 ist, gibt es nichts zu tun
	\item sucht im Stapel nach dem Blatt (= Speicher) mit der entsprechenden Nummer
	\item legt das Blatt vor sich auf das Pult
	\item quittiert die Aktion mit: \voice{Fertig!}
	\item wartet, bis der Datenbote seine Aufgabe erledigt hat
	\item legt das Blatt wieder im Stapel ab
\end{enumerate}


\vspace{1cm}
\section*{Der Datenbote}

\subsection*{Zusatzmaterial}
\begin{itemize}
	\item Schreibzeug
\end{itemize}

\subsection*{Fähigkeiten}
\begin{itemize}
	\item kann die Zahl auf dem Blatt, das auf dem Pult beim Adressierer liegt, lesen
	\item kann eine Zahl auf das Blatt beim Adressierer schreiben
	\item transportiert die gelesene resp. zu schreibende Zahl auf Befehl des CPU-Leiters
\end{itemize}

\subsection*{Aktionen}
\begin{enumerate}
	\item wartet auf den Befehl des CPU-Leiters
	\item führt den Datentransport aus
	\item falls bereits eine Zahl auf dem Blatt steht, wird diese durchgestrichen bevor die neue Zahl notiert wird
	\item quittiert die Aktion mit: \voice{Fertig!}
\end{enumerate}


\pagebreak
\section*{Der Speicherzeiger}

\subsection*{Zusatzmaterial}
\begin{itemize}
	\item Blatt \dq{}Speicherzeiger\dq{} (Register)
	\item Schreibzeug
\end{itemize}

\subsection*{Fähigkeiten}
\begin{itemize}
	\item weiss, an welcher Speicherstelle der als nächstes auszuführende Befehl steht
	\item hat ein eigenes Blatt zum Merken der aktuellen Position im Speicher
	\item kann die Zahl auf seinem Blatt auf Befehl um 1 erhöhen
	\item kann die Zahl auf seinem Blatt auf Befehl auf einen bestimmten Wert setzen
\end{itemize}

\subsection*{Aktionen}
\begin{enumerate}
	\item wartet auf den Befehl des CPU-Leiters
	\item ersetzt die Zahl auf seinem Blatt durch den mitgeteilten Wert oder erhöht die Zahl um 1
	\item quittiert die Aktion mit: \voice{Fertig!}
\end{enumerate}
oder
\begin{enumerate}
	\item wartet auf den Befehl des CPU-Leiters
	\item teilt die Zahl auf seinem Blatt dem CPU-Leiter mit
\end{enumerate}


\vspace{2cm}
\section*{Der Rechenknecht}

\subsection*{Zusatzmaterial}
\begin{itemize}
	\item 2 Blätter \dq{}Rechenknecht Blatt 1 und 2\dq{} (2 Register)
	\item Schreibzeug
\end{itemize}

\subsection*{Fähigkeiten}
\begin{itemize}
	\item hat ein Arbeitsblatt (Blatt 1)
	\item hat eine eigene Speicherstelle (Blatt 2)
	\item kann Zahlen (von Blatt 1 und Blatt 2) addieren
	\item kann Zahlen (Blatt 1 und Blatt 2) vergleichen (schreibt 01, wenn sie gleich sind, 00 sonst)
\end{itemize}

\subsection*{Aktionen}
\begin{enumerate}
	\item wartet, bis er vom CPU-Leiter einen Befehl erhält
	\item führt den Befehl aus
	\item streicht die Zahl auf Blatt 2 durch und schreibt das Resultat auf Blatt 2
	\item quittiert die Aktion mit: \voice{Fertig!}
\end{enumerate}


\pagebreak
\vspace{1cm}
\section*{Der Zeitgeber}

\subsection*{Zusatzmaterial}
\begin{itemize}
	\item Blatt \dq{}Zeitgeber\dq{} (Register)
	\item Schreibzeug
\end{itemize}

\subsection*{Fähigkeiten}
\begin{itemize}
	\item stoppt die Zeit, die für die Abarbeitung der Runden erforderlich ist und notiert sie auf ein Blatt
\end{itemize}

\subsection*{Aktionen}
\begin{enumerate}
	\item notiert die Zeit, die für die vergangene Runde benötigt wurde
	\item setzt die Stoppuhr zurück und startet sie für die Aufzeichnung der nächsten Runde
	\item quittiert die Aktion mit: \voice{Fertig!}
\end{enumerate}


\vspace{2cm}
\section*{Der Bildschirm (ist nur passiv, wird daher meist nicht als Rolle gespielt)}

\subsection*{Zusatzmaterial}
\begin{itemize}
	\item Blatt \dq{}Bildschirm\dq{}
\end{itemize}

\subsection*{Fähigkeiten}
\begin{itemize}
	\item empfängt Zahlen vom Datenboten
	\item stellt Zahlen für alle sichtbar dar
\end{itemize}

\subsection*{Aktionen}
\begin{enumerate}
	\item wartet, bis der Datenbote auf seinem Blatt etwas notiert hat
	\item hält das Blatt für die ganze Klasse sichtbar hoch (für einige Sekunden)
	\item quittiert die Aktion mit: \voice{Fertig!}
\end{enumerate}


\end{document}
